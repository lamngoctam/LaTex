
\section{INTRODUCTION}
\begin{itemize}
\color{red}
\item Novelty: Literature review
\item Goal: What question you're trying to answer
\item Motivation: Why you're asking the question
\end{itemize}

\textcolor{blue}{
\uline{Guide:} \textit{Goal: provide context and encourage reader to read the paper.\\
1. Background and motivation (1 paragraph)\\
2. Overview of the paper and contributions (1-2 paragraphs)\\
3. More details and summary of the approach\\
4. Summary of the results and conclusions}.\\
\noindent\uline{Overview}: Q4. Why should the community care?\\
\noindent\uline{Related work}: Q1. What did the community know before you did whatever you did?\\
\noindent\uline{Contribution}: Q3. Why exactly did you do?\\
We focus on....\\
We propose ABC algorithm...\\
We prove that ....\\
We demonstrate the EFG problem through x case studies (Section 3.4). We evaluate the ... (Section 4,5).\\
}

In this paper, we present discrete path planning of platonic solids including cube, tetrahedron, octahedron, icosahedron, and dodecahedron. These are types of convex polyhedra with equivalent faces constituted to congruent convex regular polygons....

Not much work has been done in path planning under considering rolling contact. [1] and [2] proposed XYZ method. In their work, they did XYZ (how they did).... However, they did not perform ABC.... $=>$ \textcolor{red}{mention Types of rolling contact, and the paper of Z.Li}

Literature in the path planning domain describes obstacles avoiding of two general types - continuous and discrete. \textbf{Continuous path planning} ....[][] \textbf{Discrete path planning} ....[][]. However, bla bla bla ...

Bla bla ....

On the other hand, bla bla bla...

Therefore, in this study, we present three cases of platonic path planning in terms of path finding for the same position and different orientation of initial configuration and goal configuration, direct searching for the long distance between two configuration, and bidirect search within obstacles.

Or: This paper presents a methodology for path planning of platonic solids in known environment. Bla bla ... ref Introduction from "Path planning in multi-scale ocean flows..."

A second contribution of this paper is a technique to compute .... 


We explain our algorithms in Section II. We go over experiments and results in simulation in Section III. We verify our algorithms by executing them on a 3D model of the Statue of David and confirming that collision-free trajectories are efficiently generated. Our primary evaluation metric is time taken for the search. We discuss the performance of each individual search, as well as the advantages and shortcomings. Finally, we discuss possible future steps for this work in Section IV.\\

