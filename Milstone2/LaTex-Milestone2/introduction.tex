
\section{INTRODUCTION}
\noindent Path planning algorithm is one of the challenging problems in nonholonomic systems which, when resolved achieves the dexterous manipulation of objects in an unknown or part-known environment. 
This problem is mainly applied to the fields of robotics, artificial intelligence and autonomous vehicles. 
In robotics, the motivation of path planning is to find a possible path from an initial configuration, avoiding the obstacles and achieving the goal configuration \cite{Zhang2018_PP_mobileRobot}. 
Based on the task of robot performance, there are mainly two kinds of planning including feasibility and optimality. 
The former is to find a plan for only achieving the path while the latter is to find an optimal path. 
In the artificial intelligence fields, searching for actions to attain the desired goal state with receiving reward is employed, including decision-theoretic methods. 
Each specific path planning algorithm is usually implemented in a parameter space, such as configuration space or free space, which generates the feasible path connecting the two given points. 
Defining the state space is also one of the important steps for planning purpose. The configuration space or C-space which includes all possible configurations in a physical system is applied for solving path planning problems in n-dimensional. 
Examples of solving the path planning problems from Lavelle \cite{LaValle06_PlanningAlgorithm} and Kavraki \cite{Kavraki96_PRM_HighDimensionSPace} presented the feasible paths avoiding obstacles in the high dimensional configuration space. \\

\noindent Rolling contact between rigid bodies has been considered as nonholonomic systems in order to solve the problem of dexterous manipulation of industrial parts. 
The goal of rolling manipulation is to roll the part from an initial configuration to the goal configuration. 
It can be divided into three types of rolling contacts including point contact \cite{Cai86_PlanarMotion_PointContact}, \cite{Cai87_SpatialMotion_PointContact}, line contact \cite{Cai88_SpatialMotion_LineContact} and surface contact \cite{Borisov08_ChaplypinBall_FixSphere}. 
A simple experiment of a rolling polyhedral part on a table, mentioned in \cite{Bicchi2004_Reachability_steering_Polyhedra}, showed that object manipulation with polyhedral surfaces without sliding can be executed by nonholonomic constraints through rolling. 
Some cases of rolling polyhedral objects through graspless manipulation have been studied in the robotics field \cite{Aiyama93_Pivoting}, \cite{Erdmann91_polyhedronRolling_on_table}. 
Due to the lack of complete research of contact kinematics and rolling manipulation with discretized objects, planning for rolling polyhedral parts under reorientation with smooth and non-smooth systems still attracts attention from the research community.\\

\noindent Planning techniques are categorized into different aspects. 
The basic idea of discrete path planning in most cases is that state-space models will be used to demonstrate the distinct situation in which the task of a planning algorithm solves the sequence actions, transforming from an initial state to other states \cite{Lavalle98rapidly_exploringrandom}. 
For example, Thomas \cite{Thomas_2003_Trajectory} applied Delaunay triangulations to discretize the environment, and cubic spline representations are proposed to meet robot kinematic constraints.
Considering other research on geometry, the paper \cite{Lamiraux_2001_Smooth_MP} studied the continuous curvature on smooth curves which has been integrated within the probabilistic approaches in order to compute the piecewise smooth paths for a car-like vehicle as a four-dimensional system. 
Whereas dealing with nonholonomic constraints, a sampling-based road map technique has been proposed in \cite{Cheng01_RRT-BasedTrajectory}, which determined trajectories and re-entry trajectories for hovercrafts and rigid spacecrafts. 
Based on decomposing space into cells \cite{Conner03_LocalFunction_Nagivation}, a potential field without local minima was assigned with polygonal partitions of planar environments to solve the Laplace’s equation problems in each presence cell. 
Applications for these techniques in discrete space is limited by a grid.\\

\noindent Not much work has been done in path planning considering rolling contact constraint. Some types of moving polyhedral parts have been investigated on a plane such as sliding on a face, tumbling through the edges or pivoting \cite{Aiyama93_Pivoting} through the vertex. 
The planning motions of rolling polyhedral parts through the edges were clearly represented in \cite{Marigo97_PolyhedraManipulation_rolling}. 
This paper presented some results about changing an orientation of a polyhedron through its edge's contact on a fixed plane without slipping.
Experimental works from the article demonstrated that the manipulation of rolling polyhedron on a plane where the set of configurations has different structures of the polyhedral parts can be reached by rolling through its edges. 
In the experimental validation, a unit cube will reach the next position by rolling  along the edges on a square mesh considering the given tolerance which leads to reaching an orientation closer. 
The paper also proposed a concept of path planning algorithm with the tolerance to achieve the goal configuration, which was considered as an important condition to generate an accurate path.
However, the practical application may not be successful on robot manipulation.\\

\noindent Marigo \cite{Marigo00_PlanningMotion_Polyhedra_Rolling} proposed the path planning for polyhedron in the case of an octahedron with eight faces rolling and translating on a plane. 
For the octahedron rolling algorithm, a list of faces containing the vertices and edges stored parts of the polyhedron. 
The defect angles are also computed between two connected faces.
The algorithm initialized a start configuration, a desired final configuration, and given a polyhedron with a set of geometrical parameters. 
The steps of planning include displacing and reorienting the polyhedral part until achieving the final configuration. Nevertheless, the algorithm may not satisfy with the accuracy for more general polyhedron.\\

\noindent Therefore, in this study, we propose a discrete path planning algorithm based on the tree exploration method for the five types of platonic solids including cube, tetrahedron, octahedron, icosahedron, and dodecahedron. The path-finding algorithm focuses on rolling a solid on the associate grid from an original position to its initial position with different orientation. 
%
The study is organized as following. Section \ref{sec:problemFormulation} covers the general properties in geometrical aspect of the five types platonic solids. Section \ref{sec:Algorithm} describes path planning algorithm based on tree exploration technique. Finally, section \ref{sec:eva} shows the results of the proposed path planning algorithm under considering their different geometrical properties, then comes the conclusion of the paper.

%
%%%================
%\begin{itemize}
%\color{red}
%\item Novelty: Literature review
%\item Goal: What question you're trying to answer
%\item Motivation: Why you're asking the question
%\end{itemize}
%
%\textcolor{blue}{
%\uline{Guide:} \textit{Goal: provide context and encourage reader to read the paper.\\
%1. Background and motivation (1 paragraph)\\
%2. Overview of the paper and contributions (1-2 paragraphs)\\
%3. More details and summary of the approach\\
%4. Summary of the results and conclusions}.\\
%\noindent\uline{Overview}: Q4. Why should the community care?\\
%\noindent\uline{Related work}: Q1. What did the community know before you did whatever you did?\\
%\noindent\uline{Contribution}: Q3. Why exactly did you do?\\
%We focus on....\\
%We propose ABC algorithm...\\
%We prove that ....\\
%We demonstrate the EFG problem through x case studies (Section 3.4). We evaluate the ... (Section 4,5).\\
%}
%
%In this paper, we present discrete path planning of platonic solids including cube, tetrahedron, octahedron, icosahedron, and dodecahedron. These are types of convex polyhedra with equivalent faces constituted to congruent convex regular polygons....
%
%Not much work has been done in path planning under considering rolling contact. [1] and [2] proposed XYZ method. In their work, they did XYZ (how they did).... However, they did not perform ABC.... $=>$ \textcolor{red}{mention Types of rolling contact, and the paper of Z.Li}
%
%Literature in the path planning domain describes obstacles avoiding of two general types - continuous and discrete. \textbf{Continuous path planning} ....[][] \textbf{Discrete path planning} ....[][]. However, bla bla bla ...
%
%Bla bla ....
%
%On the other hand, bla bla bla...
%
%Therefore, in this study, we present three cases of platonic path planning in terms of path finding for the same position and different orientation of initial configuration and goal configuration, direct searching for the long distance between two configuration, and bidirect search within obstacles.
%
%Or: This paper presents a methodology for path planning of platonic solids in known environment. Bla bla ... ref Introduction from "Path planning in multi-scale ocean flows..."
%
%A second contribution of this paper is a technique to compute .... 
%
%
%We explain our algorithms in Section II. We go over experiments and results in simulation in Section III. We verify our algorithms by executing them on a 3D model of the Statue of David and confirming that collision-free trajectories are efficiently generated. Our primary evaluation metric is time taken for the search. We discuss the performance of each individual search, as well as the advantages and shortcomings. Finally, we discuss possible future steps for this work in Section IV.\\

