
%page setting
\usepackage[utf8]{inputenc}
\usepackage[left=25mm, right=25mm, top=25mm, bottom=25mm]{geometry}


%\usepackage[colorlinks=true]{hyperref}
\usepackage{hyperref}

\usepackage{url}
\usepackage{graphicx}
\usepackage{float}
\usepackage{amsfonts}
\usepackage{amsmath}
\usepackage{amssymb}
%\usepackage{arevmath}     % For math symbols

\usepackage{enumerate}
\usepackage{fancyhdr} %footer-header

%%%%%%%%%%%-------------- page editors -----------------------
%\renewcommand{\baselinestretch}{1.5}
\pagestyle{fancy}
\fancyhead{}
%\fancyfoot{}
\renewcommand{\headrulewidth}{0pt}
\renewcommand{\footrulewidth}{1pt}

%------underline setting--------
\usepackage{ulem}
\renewcommand{\ULdepth}{1.8pt}
%\setlength{\textfloatsep}{10pt plus 1.0pt minus 15.0pt}

\setlength{\belowcaptionskip}{-10pt}


%%%%%%%%%%%--------Table-related commands------
\usepackage{array} 
%To automatically break longer lines of text within cells, define fixed-width columns
%\usepackage[table,xcdraw]{xcolor}
\usepackage[dvipsnames]{xcolor}

%\usepackage{multirow}
\usepackage{tabularx} % length of table
\usepackage{caption}  % space btw caption and table

\usepackage{booktabs}
% produce heavier lines as table frame (\toprule, \bottomrule) and lighter lines within a table (\midrule).

%link: https://texblog.org/2017/02/06/proper-tables-with-latex/
\newcolumntype{V}{>{\bf\centering\arraybackslash} m{0.2\linewidth} } %Repeat column type
%------------------
\usepackage{stackengine}

\usepackage{makecell,interfaces-makecell}
\usepackage{tabu,stackengine}

%%%%%%%%%%%--------------Tikz-------------------------------
\usepackage{import}
\usepackage{tikz}
\usepackage{tikz-3dplot}
\usepackage{subfigure}

\usepackage[edges]{forest}
\usetikzlibrary{arrows.meta}
\usepackage{adjustbox}
\usetikzlibrary{fit}


\usetikzlibrary{trees,snakes}

\usepackage{pdfpages}

\usetikzlibrary{shapes.geometric} % draw the flow chart
\usetikzlibrary{positioning} 
% https://tex.stackexchange.com/questions/94386/package-pgf-math-error-unknown-operator-o-or-of
\usetikzlibrary{automata}         % for graph-automata
\usepackage{pgfplots}             % for plotting data
%-----------------------------------------------
%         PACKAGES for \tkzDefPoints,\tkzPolygon
%-----------------------------------------------
\usepackage{tkz-euclide} % will load tkz-base
\usepackage{siunitx} 	 % to display angle in degree \ang{180} or \usepackage{gensymb}
\usetkzobj{all}			 % loads all objects used by tkz-euclide

%%%%%%%%%%\usepackage{verbatim} 
% a  drawing of a tetrahedron inscibed in a parallelepipe from https://www.overleaf.com/project/5dc97a9a3af030000156a35f

%%---------- Algorithm---------
%\usepackage[ruled,vlined]{algorithm2e}

\usepackage{algorithm}
%\usepackage{caption}
\usepackage{algpseudocode}


% https://gking.harvard.edu/files/natnotes2.pdf  -> # types of natbib


%-------- for biber reference
%\usepackage{pdfpages}
\usepackage[style=nature,backend=biber]{biblatex}


%\usepackage[backend=biber,style=nature,citestyle=numeric-comp,sorting=ynt,defernumbers=true]{biblatex}
\addbibresource{citation.bib}

%----------------


