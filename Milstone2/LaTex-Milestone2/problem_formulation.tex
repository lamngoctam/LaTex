\section{PROBLEM FORMULATION}
\label{sec:problemFormulation}
\noindent\uline{Five types of platonic solids}:
The platonic solids are also called regular polyhedra have the convex polyhedra properties. There are only five solids namely cube, tetrahedron, octahedron, dodecahedron and icosahedron. Some of the equivalent statements are used to describe the platonic solids including all the vertices lie on a sphere, all the dihedral angle are equal, and all solid angles are equivalent.\\

\begin{figure}[h]
\centering
	\includegraphics[width=1\textwidth]{image/5Platonic1.png}
	\caption{Platonic solids. From left to right: Tetrahedron, Cube, Octahedron, Dodecahedron, and Icosahedron}
	\label{fig:platonicSolids}
\end{figure}
%
% 
%
%
%
\noindent \uline{Geometrical parameters}: 
The number of faces, edges, and vertices of each types of platonic solids is described in the Table \ref{tab:tb1}. In mathematics, the Euler's formula shows the relationship between total number of vertices ($V$), edges ($E$), and faces ($F$) by the equation $V-E+F=2$. The icosahedron has the largest number of faces with $20$ while the tetrahedron has only $4$ faces. The number of faces is high the close to sphere is approximate.
%
%
%
The Table \ref{tab:tb2} shows the radii of each solids with inradius ($r_i$), midradius ($\rho$) and circumradius ($R$).\\

\begin{table}[h]
\centering
\caption{Properties of polyhedron}
\label{tab:tb1}
\begin{tabular}{|l|c|c|c|c|c|}
\hline
             & Faces & Edges & Vertices & Edges on each face & Edges meeting at each vertices \\ \hline
Tetrahedron  & 4     & 6     & 4        & 3                  & 3                            \\ \hline
Cube         & 6     & 12    & 8        & 4                  & 3                            \\ \hline
Octahedron   & 8     & 12    & 6        & 3                  & 4                            \\ \hline
Dodecahedron & 12    & 30    & 20       & 5                  & 3                            \\ \hline
Icosahedron  & 20    & 30    & 12       & 3                  & 5                            \\ \hline
\end{tabular}
\end{table}
% generate the table from https://www.tablesgenerator.com/#
%
%
% 
%
%
%
Here is your table \ref{tab:tb2}

\begin{table}[h]
\centering
\caption{Geometrical parameters of platonic solids}
\label{tab:tb2}
\begin{tabular}{|l|c|c|c|c|}
\hline
             & $r_d$	                             & $\rho$                    & R	     					      & dihedral angles ($\beta$)	\\ \hline
Tetrahedron  & $\frac{1}{12}\sqrt{6}$    			 & $\frac{1}{4}\sqrt{2}$     & $\frac{1}{4}\sqrt{6}$              & $\cos^{-1}(\frac{1}{3})$                       \\ \hline
Cube         & $\frac{1}{2}$                         & $\frac{1}{2}\sqrt{2}$     & $\frac{1}{2}\sqrt{3}$              & $\frac{1}{2}\pi$                \\ \hline
Octahedron   & $\frac{1}{6}\sqrt{6}$    			 & $\frac{1}{2}$    	     & $\frac{1}{2}\sqrt{2}$      		  & $\cos^{-1}(-\frac{1}{3})$               \\ \hline
Dodecahedron & $\frac{1}{20}\sqrt{250+110\sqrt{5}}$  & $\frac{1}{4}(3+\sqrt{5})$ & $\frac{1}{4}(\sqrt{15}+\sqrt{3})$  & $\cos^{-1}(-\frac{1}{5}\sqrt{5})$              \\ \hline
Icosahedron  & $\frac{1}{12}(3\sqrt{3}+\sqrt{15})$   & $\frac{1}{4}(1+\sqrt{5})$  & $\frac{1}{4}\sqrt{10+2\sqrt{5}}$  & $\cos^{-1}(-\frac{1}{3}\sqrt{5})$               \\ \hline
\end{tabular}
\end{table}

\clearpage
\newpage
%\vspace{1in}
%\begin{table}%[h!]
%	\centering
%	\caption{Optimized Parameter Values}
%	\begin{tabular}{llllp{7em}p{7em}l}
%		\toprule
%		Type of Controller & Parameter    & Xmax & Xmin & \raggedright Iter. reqd. for convergence & Optimized value & $W_\mathrm{min}$ \\ \midrule
%			PSO-SOSMC        & $c_1$        & 5    & 0.1  & 37                                       & 4.75            & 68.43 \\
%			                 & $c_2$        & 5    & 0.1  & 10                                       & 4.273           & 20.45 \\
%			                 & $\lambda_1 $ & 5    & 0.1  & 37                                       & 2.75            & 68.43\\
%			                 & $\lambda_2 $ & 5    & 0.1  & 10                                       & 3.59            & 20.45\\
%			                 & $W_1 $       & 1    & 0.05 & 37                                       & 0.43            & 68.45\\
%			                 & $W_2 $       & 1    & 0.05 & 10                                       & 0.218           & 20.43\\ \cmidrule(lr){2-7}
%			PSO-BELBIC       & $W_1$        & 5    & 0.1  & 36                                       & 4.5             & 27.34 \\
%			                 & $W_2$        & 5    & 0.1  & 14                                       & 4.5             & 61.63 \\
%			                 & $G_1$        & 5    & 0.1  & 36                                       & 1.4             & 27.34 \\
%			                 & $G_2$        & 5    & 0.1  & 14                                       & 1.4             & 61.63\\ \bottomrule
%		\end{tabular}
%\end{table}