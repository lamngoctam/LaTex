\section{MODEL DESCRIPTION}
\label{sec:problemFormulation}
\noindent\uline{Platonic solids}:
The platonic solids are also called regular polyhedra have the convex polyhedra properties. 
There are exactly five regular polyhedra namely cube, tetrahedron, octahedron, dodecahedron and icosahedron (Figure \ref{fig:platonicSolids}). 
Some of the equivalent statements are used to describe the platonic solids, including all the vertices lie on a sphere, all the dihedral angle are equal, and all solid angles are equivalent. 
The tetrahedron is folded by 4-sided pyramid, the octahedron has the double-pyramid with $8$ faces and 20-sided pyramid for the icosahedron. The cube is constructed by $6$ square faces while the dodecahedron is composed of 12-sided of regular pentagons.\\

\begin{figure}[h]
\centering
	\includegraphics[width=0.9\textwidth]{image/5Platonic1.png}
	\caption{The platonic solids. From left to right with models and unfolding models: the tetrahedron, the cube, the octahedron, the dodecahedron, and the icosahedron}
	\label{fig:platonicSolids}
\end{figure}
%
% 
%
%
%
\noindent \uline{Geometrical parameters}: 
The number of faces, edges, and vertices of each type of platonic solids is described in the Table \ref{tab:tb1}. 
In mathematics, the Euler's formula shows the relationship between total number of vertices ($V$), edges ($E$), and faces ($F$) by the Eq. \ref{equa:eq1}
%
\begin{equation} 
\label{equa:eq1}
\begin{split}
V-E+F=2
\end{split}
\end{equation}
% 
The icosahedron has the largest number of faces with $20$ while the tetrahedron has only $4$ faces. 
If each face of platonic solids has $i$ sides and $k$ edges of the polyhedron meet at each vertex.
The conditions are $i,k$ greater than $3$ ($i,k\geq3$) because every face have at least three edges and at least three edges of faces meet at each vertex.
The faces are equilateral triangle if $i=3$ and changing the values of $k$ will yield the other three types of polyhedron, including the tetrahedron ($k=3$), the octahedron ($k=4$),and the icosahedron ($k=5$). When $i=4$, the faces are square and the cube has $k=3$ edges which meet at each vertex. The last case with $i=4$ of regular pentagons and only $k=3$ will generate the dodecahedron. \\

% %Check p.61 Euler's gem book
%
\begin{table}[H]
\centering
\caption{Properties of polyhedron}
\label{tab:tb1}
\begin{tabular}{|l|c|c|c|c|c|}
\hline
             & Faces & Edges & Vertices & Edges on each face & Edges meeting at each vertex \\ \hline
Tetrahedron  & 4     & 6     & 4        & 3                  & 3                            \\ \hline
Cube         & 6     & 12    & 8        & 4                  & 3                            \\ \hline
Octahedron   & 8     & 12    & 6        & 3                  & 4                            \\ \hline
Dodecahedron & 12    & 30    & 20       & 5                  & 3                            \\ \hline
Icosahedron  & 20    & 30    & 12       & 3                  & 5                            \\ \hline
\end{tabular}
\end{table}
% generate the table from https://www.tablesgenerator.com/#
%
% %Check p.61 Euler's gem book

\noindent The path planning for platonic solids focuses on the rolling of the models through edge-contact. Each edge is shared by two faces and each face may has $e$ edges. Assume that $\Delta $ is the quantity faces which contact at an edge or $E=\frac{1}{2}\Delta $. However, each vertex will be shared by $f$ faces. Then, $V=\frac{\Delta}{f}$. After substituting these two quantities into the Eq. \ref{equa:eq1}, the result is:
%
%
\begin{equation} 
\label{equa:eq2}
\begin{split}
V-E+F &= 2\\
\frac{\Delta}{f} - \frac{1}{2}\Delta + F &= 2\\
\rightarrow F &= \frac{4f}{2e-fe+2f}\\
\end{split}
\end{equation}
%

\noindent From the Eq. \ref{equa:eq2}, the condition is that $2e-fe+2f$ should positive because $F$ and $4f$ are both positive. 
As above requirements with ($e\geq3$) and ($f\geq3$), there are only five pairs of integers ($e,f$) satisfy this condition. 
So, the results of these pairs are shown in the last two columns in the Figure \ref{tab:tb1}.\\

\noindent In order to execute the transformation stage, a rotation angle needs to determine in the path planning algorithm.
In the context of regular convex polyhedra, a rotation angle is supplementary with a dihedral angle which is the angle between two connected faces along an edge inside the polyhedra.
The Table \ref{tab:tb2} shows the radii of each solid with the inradius ($r_i$), the midradius ($\rho$), the circumradius ($R$) and the dihedral angles ($\beta$).\\

\begin{table}[h]
\centering
\caption{Geometrical parameters of platonic solids}
\label{tab:tb2}
\begin{tabular}{|l|c|c|c|c|}
\hline
             & $r_d$	                             & $\rho$                    & R	     					      & dihedral angles ($\beta$)	\\ \hline
Tetrahedron  & $\frac{1}{12}\sqrt{6}$    			 & $\frac{1}{4}\sqrt{2}$     & $\frac{1}{4}\sqrt{6}$              & $\cos^{-1}(\frac{1}{3})$                       \\ \hline
Cube         & $\frac{1}{2}$                         & $\frac{1}{2}\sqrt{2}$     & $\frac{1}{2}\sqrt{3}$              & $\frac{1}{2}\pi$                \\ \hline
Octahedron   & $\frac{1}{6}\sqrt{6}$    			 & $\frac{1}{2}$    	     & $\frac{1}{2}\sqrt{2}$      		  & $\cos^{-1}(-\frac{1}{3})$               \\ \hline
Dodecahedron & $\frac{1}{20}\sqrt{250+110\sqrt{5}}$  & $\frac{1}{4}(3+\sqrt{5})$ & $\frac{1}{4}(\sqrt{15}+\sqrt{3})$  & $\cos^{-1}(-\frac{1}{5}\sqrt{5})$              \\ \hline
Icosahedron  & $\frac{1}{12}(3\sqrt{3}+\sqrt{15})$   & $\frac{1}{4}(1+\sqrt{5})$  & $\frac{1}{4}\sqrt{10+2\sqrt{5}}$  & $\cos^{-1}(-\frac{1}{3}\sqrt{5})$               \\ \hline
\end{tabular}
\end{table}

\clearpage
\newpage
%\vspace{1in}
%\begin{table}%[h!]
%	\centering
%	\caption{Optimized Parameter Values}
%	\begin{tabular}{llllp{7em}p{7em}l}
%		\toprule
%		Type of Controller & Parameter    & Xmax & Xmin & \raggedright Iter. reqd. for convergence & Optimized value & $W_\mathrm{min}$ \\ \midrule
%			PSO-SOSMC        & $c_1$        & 5    & 0.1  & 37                                       & 4.75            & 68.43 \\
%			                 & $c_2$        & 5    & 0.1  & 10                                       & 4.273           & 20.45 \\
%			                 & $\lambda_1 $ & 5    & 0.1  & 37                                       & 2.75            & 68.43\\
%			                 & $\lambda_2 $ & 5    & 0.1  & 10                                       & 3.59            & 20.45\\
%			                 & $W_1 $       & 1    & 0.05 & 37                                       & 0.43            & 68.45\\
%			                 & $W_2 $       & 1    & 0.05 & 10                                       & 0.218           & 20.43\\ \cmidrule(lr){2-7}
%			PSO-BELBIC       & $W_1$        & 5    & 0.1  & 36                                       & 4.5             & 27.34 \\
%			                 & $W_2$        & 5    & 0.1  & 14                                       & 4.5             & 61.63 \\
%			                 & $G_1$        & 5    & 0.1  & 36                                       & 1.4             & 27.34 \\
%			                 & $G_2$        & 5    & 0.1  & 14                                       & 1.4             & 61.63\\ \bottomrule
%		\end{tabular}
%\end{table}