\section{EVALUATION}
\label{sec:eva}
The proposed algorithm for platonic solids path planning by rolling through edge contact was implemented in MATLAB environment. 
In general of path planning, there are three case studies including same location and different orientation between initial configuration and goal configuration, long distance between two configurations, and bi-direction path finding. 
To validate the proposed algorithm, this study only considers the first case study of path planning that both initial and goal configuration have the same positions and different orientations.\\  

\noindent Assume that the platonic solids' edges has the same length with $a$ ($a=1$) and one of the faces of the platonic solids contact to $OXY$. 
The environment for each type of platonic solids is discretized from the smooth surface. 
For example, cube solid will roll on the square grid while tetrahedron, octahedron, and icosahedron solids roll on the triangle gird. 
Only dodecahedron solid rolls on the pentagon grid with the two specific cases including the gaps and overlaps between pentagons. 
The path planning based on rolling for the five cases of platonic solids is represented as the following. 

\subsection{Cube solid}
\noindent\uline{Properties}:
The cube has a length $a$ which is the same as the length of side of each grid square.
Path planning for rolling a cube is not complex as other cases of platonic solids. 
The rolling angle $\beta$ has the value of $\frac{\pi}{2}$ as shown in the Figure \ref{fig:cubeGeo1} ($\beta=\angle{IKI_1}=\frac{\pi}{2}$). 
The center of cube will move along the curve $II_2I_1$ following the $Ox$ axis. 
The proposed algorithms will focus on rolling along $Ox$ or $Oy$ axis. 
However, all the vertices are stored in a matrix will change their coordinates in $3D$ space. 
An example from the Figure \ref{fig:cubeGeo1}, eight vertices of the cube $ABCDMNOP$ and the center $I$ are stored in a matrix $M(9,3)$ with $9$ rows for nine $3D$ points, and $3$ columns for $3D$ coordinate $Ox,Oy,Oz$ of each point. 
After finishing the rotation angle $\beta$ based on the rotation axis $MN$, the coordinates of the cube's vertices are updated within a new cube $MA'D'PN'B'C'O'$. 
This is the basic step of rolling a cube through a line contact.

\begin{figure}[H]
\centering
	\includegraphics[width=1\textwidth]{image/cubeGeo1.png}
	\caption{Rotation angle of a cube on a plane}
	\label{fig:cubeGeo1}
\end{figure}

\noindent\uline{Path planning}: 
The only way to move from initial position to goal position of the cube solid is by rolling from square to square on the grid without moving diagonal. 
The Figure \ref{fig:cubePath0} shows the first three layers of cube path finding on the grid. 
The algorithm implements in $O(|E|^3)$ running time from the second layer called the expansion three branches from a tree. 
All $3D$ coordinates of the cube are stored in a matrix which can affect to the computer's storage capacity. 
When the updated cubes achieved one of the same previous configurations, the execution time can be reduced by releasing this configuration or stopping the expansion of this tree branch.
The $*$ position in the grid (Figure \ref{fig:cubePath0}) is occupied by the two updated cubes with different orientations in the $Layer\ 2$. 

\begin{figure}[H]
\centering
	\includegraphics[width=1\textwidth]{image/cubePath00.png}
	\caption{The first four layers of cube rolling}
	\label{fig:cubePath0}
\end{figure}

\noindent To be more visualized, the Figure \ref{fig:Cube1Case1} indicates the first step of rolling of the cube with its coordinates in the red, green and blue arrows. The first shortest path of cube rolling are shown in the Figure \ref{fig:Cube2Case1}. 

\begin{center}
\begin{figure}[H]
\subfigure[The initial configuration is the same position but different orientation with goal configuration]{
	\includegraphics[width=0.5\textwidth]{image/cube11.jpg}
	\label{fig:Cube1Case1}
	}
\hfill
%\subfigure[First four paths of the cube rolling]{	
%	\includegraphics[width=0.5\textwidth]{image/cubePath4Dirs.jpg}
\subfigure[Shortest path of cube rolling]{
\includegraphics[width=0.5\textwidth]{image/cubePath1.pdf}
%    \includegraphics[page=2,width=.5\textwidth]{image/test2.pdf}	
	\label{fig:Cube2Case1}
	}
\caption{Initial configuration and shortest path of cube rolling}
\end{figure}
\end{center}

%%\noindent\uline{Result}: 
%\begin{figure}[h]
%\centering
%	\includegraphics[width=0.5\textwidth]{image/cubePath1.pdf}
%%	\includepdf[pages=-,pagecommand={},width=0.5\textwidth]{image/cubePath1.pdf}
%	\caption{Shortest path of cube rolling}
%	\label{fig:cubePath1}
%\end{figure}

%\begin{figure}[h]
%	\centering
%		\begin{subfigure}[t]
%			\includegraphics[width=0.5\textwidth]{image/cubePathCase2Initial.jpg}
%			\subcaption{Long distance between two configurations}
%			\label{fig:Cube1Case2}
%		\end{subfigure}
%%\hfill
%		\begin{subfigure}[t]
%			\includegraphics[width=0.5\textwidth]{image/cubePathCase2DirecRolling.jpg}
%			\subcaption{Directly rolling from initial configuration to goal configuration}
%			\label{fig:Cube2Case2}
%		\end{subfigure}
%\end{figure}

%%
%%
%%
%\clearpage
%\newpage
%\noindent Although considering the case study within \\ 


\noindent \uline{Extension case}: The proposed algorithm in this study can be applied for the case study of long distance between the initial and goal configurations. 
Figure \ref{fig:cubeLongDist} shows an example with two different paths.
Shortest path-finding algorithm is added to the original algorithm to find the shortest path from start point to the goal point. 
After finishing this step, the cube updated to a new orientation with different orientation at the goal configuration. Then, the original algorithm will be implemented from the updated cube to the goal configuration.
Assume that the initial configuration is at $S_1$ and the goal configuration is at $S2$. In the first step, the red line segment $S_1S_2$ shows the shortest distance from start position to goal position. The cube will roll through this line segment and achieve the updated orientation at the goal position called shortest path for the case of long distance. 

%\begin{center}
\begin{figure}[H]
\centering
\subfigure[Path1]{
	\includegraphics[width=0.75\textwidth]{image/cubeCase2Path1.jpg}
	\label{fig:cubeLongDistPath1}
	}
\hfill
\subfigure[Path2]{
	\includegraphics[width=0.75\textwidth]{image/cubeCase2Path2.jpg}
	\label{fig:cubeLongDistPath2}
	}
\caption{The case study of long distance between the initial and goal configurations}
\label{fig:cubeLongDist}
\end{figure}
%\end{center}
%%
%%
%%
%%
%%==================================================================================
%%                               Tetrahedron solid
%%==================================================================================
%%
%%
%%
\clearpage
\newpage
\subsection{Tetrahedron solid}
\noindent\uline{Properties} 
As can be seen from the Figure \ref{fig:tetraGeo1}, the Tetrahedron has constructed by four faces of the equilateral triangles. Then the height of triangle $ABC$ is $AM$ and $AM=DM=a\sqrt{3}/2$.
Because of $r=OH=a\sqrt{6}/12$ (the radius of insphere) and $R=OA=a\sqrt{6}/4$ (the radius of circumsphere), the height of tetrahedron is $AH=OA+OH=a\sqrt{6}/3$. The rotation angle is $\beta$ determined by supplementary angles $\alpha=\arctan(2\sqrt{2})$ or $\beta=\pi-\alpha = \pi-\arctan(2\sqrt{2})$.
 
\begin{figure}[h]
\centering
	\includegraphics[width=\textwidth]{image/TetraGeo11.png}
%	\includepdf[pages=-,pagecommand={},width=0.5\textwidth]{image/cubePath1.pdf}
	\caption{Tetrahedron geometrical properties}
	\label{fig:tetraGeo1}
\end{figure}

\noindent\uline{Path Planning}:
The case study of tetrahedron in this study is the same of the rolling cube path finding which only considering the path planning through rolling from initial configuration to origin coordinate with different orientation. 
The Figure \ref{fig:TetraPathFiding} illustrates two of four cases of the tetrahedron path-finding within rolling (red and cyan arrows are pointing down to plane respectively). Tetrahedron has symmetry properties with indistinguishable for any two faces, edges and vertices. To be more specific, dihedral triangles have same three angles within $\ang{60}$. 
In one cycle of rolling a tetrahedron around any vertices, the tetrahedron always achieve the initial configuration due to six times of rolling ($6*\ang{60}=\ang{360}$, a full circle). \\

\begin{center}
\begin{figure}[h]
\subfigure[Tetrahedron path 1]{
	\includegraphics[width=0.5\textwidth]{image/tetraPath2.pdf}
	\label{fig:Tetra1Case1}
	}
\hfill
\subfigure[Tetrahedron path 2]{
	\includegraphics[width=0.5\textwidth]{image/tetraPath1.pdf}
	\label{fig:Tetra2Case1}
	}
\caption{The two of four cases}
\label{fig:TetraPathFiding}
\end{figure}
\end{center}
%%
%%
%%
%%==================================================================================
%%                              Octahedron solid
%%==================================================================================
%%
%%
%%
\clearpage
\newpage
\subsection{Octahedron solid}
\noindent\uline{Properties}: 
Dennis also went his own way and divided the sides of the triangles into equal-angles (as measured from the center of the geodesic), instead of equal-length pieces. This technique is slightly more effective at evenly distributing the triangles across the surface of the sphere. For example, compare an octahedron subdivided with frequency 20, using the linear technique (as outlined by the quiz) versus the angular technique Dennis used in this picture. 

\begin{figure}[h]
\centering
	\includegraphics[width=\textwidth]{image/octaGeo11.png}
%	\includepdf[pages=-,pagecommand={},width=0.5\textwidth]{image/cubePath1.pdf}
	\caption{Octahedron geometrical properties}
	\label{fig:octaGeo1}
\end{figure}


%\begin{wrapfigure}{H}{0.25\textwidth} %this figure will be at the right
%    \centering
%    \includegraphics[width=0.25\textwidth]{image/octaGeo2.png}
%\end{wrapfigure}

\noindent From Figure \ref{fig:octaGeo1} we have:

\begin{equation*} 
\label{octa:eq0}
\begin{split}
AK & = 2OK = 2\sqrt{(MK^2-MO^2)} = a\sqrt{2}\\
\Rightarrow OK & = OM = ON = a\frac{\sqrt{2}}{2}\\
\end{split}
\end{equation*}

\noindent To find the rotation angle, the supplementary angle should be calculated first. 

\begin{equation*} 
\label{octa:eq2}
\begin{split}
\alpha & = \angle OPK = \arctan{\frac{OK}{OP}} = \arctan{\sqrt{2}}
\end{split}
\end{equation*}

\noindent Then, the rotation angle has the result as  $\beta = \pi-2\alpha = \pi-2\arctan{\sqrt{2}}$, \\

\noindent Due to $\angle POK=\ang{90}$, we have

\begin{equation*} 
\label{octa:eq3}
\begin{split}
\frac{1}{OE^2} & = \frac{1}{OP^2}+\frac{1}{OK^2}\\
			   & = \frac{1}{(\frac{\sqrt{2}}{2})^2}+\frac{1}{(\frac{1}{2})^2}\\
\Rightarrow OE & = \frac{\sqrt{6}}{6}
\end{split}
\end{equation*}

\noindent Applying the theory of the equilateral triangle

\begin{equation*} 
\label{octa:eq4}
\begin{split}
EO & = OF = a\frac{\sqrt{6}}{6}\\
KE & = FA = a\frac{\sqrt{3}}{3}\\
EP & = FQ = a\frac{\sqrt{3}}{6}
\end{split}
\end{equation*}

\noindent\uline{Path planning}:
Based on the classical path planning which is the movement from point to other points, the octahedron path planning within rolling has three directions to move. In the Figure \ref{fig:octaGeo1}, the bottom layer $\triangle ABC$ which contacts to $OXY$ can roll with the directions $FQ, FO_1, FR$. The distance between two shortest positions is $2FR= 2a\frac{\sqrt{3}}{6}=a\frac{\sqrt{3}}{3}$. 

\noindent As can be seen from the Figure \ref{fig:octaPath1}, the shortest path within red line includes ten line segments for rolling the octahedron. The total length of the path equals $10a\frac{\sqrt{3}}{3}$ (edge length of the octahedron is $a$).

\begin{figure}[h!]
\subfigure[The first shortest path of Octahedron path rolling]{
	\includegraphics[width=0.5\textwidth]{image/octoPath1.pdf}
	\label{fig:octaPath1}
	}
\hfill
\subfigure[The second shortest path of Octahedron path rolling]{
	\includegraphics[width=0.5\textwidth]{image/octoPath2.pdf}
	\label{fig:octaPath2}
	}
\subfigure[The second shortest path of Octahedron path rolling]{
	\includegraphics[width=0.5\textwidth]{image/octoPath2.pdf}
	\label{fig:octaPath2}
	}
\caption{Three shortest paths of octahedron based rolling}
\label{fig:octaPaths}
\end{figure}

%%
%%
%%==================================================================================
%%                       Icosahedron solid
%%==================================================================================
%%
%%
%%
\clearpage
\newpage
\subsection{Icosahedron solid}
\noindent\uline{Properties}:
The convex regular icosahedron in the Figure \ref{fig:icosaGeo2} is one of the five regular Platonic solids has 12 vertices, 20 triangular faces, and 30 edges. 
Assume that the icosahedron has the edge length with $a$. 
The crossing surface of the solid at vertex $A$ perpendicular with $OD$ will generate a pentagon $ABCEF$.

\begin{figure}[h]
\centering
	\includegraphics[width=\textwidth]{image/icosaGeo22.png}
%	\includepdf[pages=-,pagecommand={},width=0.5\textwidth]{image/cubePath1.pdf}
	\caption{Icosahedron geometrical properties}
	\label{fig:icosaGeo2}
\end{figure}

\noindent At the right side of  the Figure \ref{fig:icosaGeo2}, the pentagon $ABCEF$ has circumcircle radius $R$, inscribed circle $r_i$, and the height $d_{CS}$ ($R+r_i$).
The golden ratio $\Phi$ (irrational number) has the value of $\frac{1+\sqrt{5}}{2}$ can be found by:
\begin{equation*} 
\label{icosa:eq1}
\begin{split}
\Phi & = \frac{OX}{XB} = \frac{OX}{\frac{1}{2}a}\\
\end{split}
\end{equation*}

%\sin(\angle XOB) & = \frac{XB}{OB} = \frac{\frac{1}{2}a}{\frac{\sqrt{\Phi^2+1}}{2}a}\\
%				 & = \frac{1}{\sqrt{\Phi^2 + 1}}\\
%\Rightarrow \angle{XOB} & = \arcsin{\frac{1}{\sqrt{\Phi^2 + 1}}}\\
%\Rightarrow \angle{DOB} & = 2\angle{XOB} = 2\arcsin{\frac{1}{\sqrt{\Phi^2 + 1}}}\\

\noindent Same as the case of cube's path planning algorithm, the initial step of rolling icosahedron is to find the rotation angle. 
The rotation angle is a supplementary of the $\angle{QRS}$. 
The line $OR$ is perpendicular to $QS$ and $OS = OQ = OM = r_i = \frac{\Phi^2}{2\sqrt{3}}a$. 
$BI$ is a diagonal of pentagon $BCLIJ$ and $\angle{QRS}=2\angle{ORS}$.\\

\noindent Otherwise, $OR=OX=\frac{1}{2}a\Phi$, and

\begin{equation*} 
\label{icosa:eq2}
\begin{split}
\angle{QRS} & = 2\angle{ORS}\\
\sin{\angle{ORS}} & = \frac{OS}{OR} = \frac{ \frac{\Phi^2}{2\sqrt{3}}a}{\frac{1}{2}a\Phi} = \frac{\Phi}{\sqrt{3}}\\
\Rightarrow \angle{QRS} & =2\arcsin{\frac{\Phi}{\sqrt{3}}}\\
\end{split}
\end{equation*}

\noindent To be more detail in the Figure \ref{fig:icosaGeo5}, the rotation angle of icosahedron can be determined from the result of $\angle{QRS}$, as the following. 

\begin{equation*} 
\label{icosa:eq3}
\begin{split}
\beta & = \pi- \angle{QRS}\\
      & = \pi - 2\arcsin{\frac{\Phi}{\sqrt{3}}}\\
      & = \pi- \arccos{-\frac{\sqrt{5}}{3}}\\
\end{split}
\end{equation*}

\noindent To calculate the rotation axis for all the cases of the triangular surface contact, there are three axis such as $IJ$, $JK$, and $IK$ with the $3D$ coordinates as following.

%\begin{equation*} 
\begin{align*}
\label{icosa:eq4}
JK & = [a\quad  0\quad  0]\\
IJ & = [-\frac{1}{2}a\quad  (\frac{\sqrt{3}}{6}+\frac{\sqrt{3}}{3})a\quad  0]\\
IK & = [-\frac{1}{2}a\quad  -(\frac{\sqrt{3}}{6}+\frac{\sqrt{3}}{3})a\quad  0]
\end{align*}
%\end{equation*}

\begin{figure}[h]
\centering
	\includegraphics[width=0.5\textwidth]{image/icosaGeo55.png}
	\caption{Rotation angle and rotation axis}
	\label{fig:icosaGeo5}
\end{figure}

%
% ===================================
%
\noindent\uline{Path planning} 
Path planning of the regular icosahedron through rolling in known environment has initial coordinate at $[6.5, 5.5, 0]$ and the surface contact with red arrow points to bottom. 
The goal configuration with same as the initial coordinate has the contact surface with black arrow as shown in the Figure \ref{fig:icosaPath1}. 
This figure shows the shortest path with the red line including 14 line segments. 
It means that the icosahedron rolled 14 times from the initial configuration to the goal configuration. 
The path can be seen more precisely from the top view in the Figure \ref{fig:icosaPath2}.
The total length of this icosahedron path equals $14a\frac{\sqrt{3}}{3}$ (edge length of the octahedron is $a$).

\begin{center}
\begin{figure}[h]
\subfigure[The first shortest path of Icosahedron path rolling (Red line)]{
	\includegraphics[width=0.5\textwidth]{image/icosaPath.pdf}
	\label{fig:icosaPath1}
	}
\hfill
\subfigure[The top view of path planning for Icosahedron]{
	\includegraphics[width=0.5\textwidth]{image/icosaPathTopView.pdf}
	\label{fig:icosaPath2}
	}
\caption{Path of icosahedron rolling though edges}
\label{fig:icosaPaths}
\end{figure}
\end{center}
%
%
%%==================================================================================
%%                       Dodecahedron solid
%%==================================================================================
%%
%%
%%
%
\clearpage
\newpage
\subsection{Dodecahedron solid}

%\noindent\uline{Result}: 
%\begin{figure}[h]
%\centering
%	\includegraphics[width=1\textwidth]{image/dodecahedron1.png}
%	\caption{Shortest path of cube rolling}
%	\label{fig:dodecahedron1}
%\end{figure}

\noindent\uline{Properties}: 
An dodecahedron is constructed from twelve self-intersecting faces with the same pentagon shape. These faces meet at each vertex and there are total of twenty vertices in a dodecahedron as shown in Figure \ref{fig:dodecahedron2}.
%
From the Eq. \ref{equa:eq1}, the great dodecahedron satisfy the Euler's formula ($V=20, F=12, E=30$).
%
It will be assumed that the coordinates $Oxyz$ lie on $ABCDE$ surface within $Oy$ through $A$ and $Oz$ perpendicular to $ABCDE$.
%
The 30 edges have the same length as $a$. It should be determined all the vertices' coordinates in the three dimensional system.
%
The Figure \ref{fig:dodecahedron2} indicates the lengths of each vertex from $l_1$ to $l_4$ and the angles $\alpha_1$ to $\alpha_4$ which correspond to the five sides of a pentagon.

\begin{figure}[h]
\centering
	\includegraphics[width=1\textwidth]{image/dodecahedron2.pdf}
	\caption{Dodecahedron's vertices.}
	\label{fig:dodecahedron2}
\end{figure}

\noindent The path planning will implement on a surface but it will be considered in $3D$ spaces. 
Then, each vertex will be determined on $3D$ coordinates such as the vertex $A$ has coordinate with $[A_x A_y A_z]$. 
Based on the properties of pentagon, the angle $\alpha_1=\frac{2\pi}{5}$ and $\alpha_4=\frac{\pi}{5}$, 
because the angle between $Ox$ and $Oy$ is $\frac{\pi}{2}$, the sum of $\alpha_1$ and $\alpha_2$ is $\alpha_1 + \alpha_2 = \frac{\pi}{2}$. 
Then the other two angles $\alpha_2$ and $\alpha_3$ can be founded as below. 
\begin{equation*} 
\label{equa:eq0}
\begin{split}
\alpha_2 &= \frac{\pi}{2}-\alpha_1 = \frac{\pi}{2}-\frac{2\pi}{5} = \frac{\pi}{10}\\ 
\alpha_3 &= \alpha_1-\alpha_2 = \frac{2\pi}{5}-\frac{pi}{10} = \frac{3\pi}{10}
\end{split}
\end{equation*}

\noindent From the Figure \ref{fig:dodecahedron2}(c), these labelled dimensions can be calculated as $l_1 = R_d = \frac{a}{2\sin{\alpha_4}}$ with $R_d$ is the circumradius of dodecahedron, $l_2 = l_1\cos{\alpha_4}$, $l_3 = l_1\cos{\alpha_2}$, and $l_4 = l_1\sin{\alpha_2}$. 
The golden ratio $\Phi$ with the value of $\frac{1+\sqrt{5}}{2}$ which is the length of the diagonal of a square with $1$ unit length of sides is used to calculate the circumscribed sphere radius of dodecahedron.
Assume that the dodecahedron has the length $a$, the radius of an inscribed sphere is $r_i = \frac{a}{20}\sqrt{10(25+11\sqrt{5})}$ and the circumscribed sphere radius is $r = a\frac{\sqrt{3}}{2}\frac{1+\sqrt{5}}{2}$.\\


\noindent In the path-finding through rolling, the proposed algorithm focuses on the transformation and translation of $20$ vertices of a dodecahedron. 
Although rolling dodecahedron's faces on $2D$ surface, the bottom surface which integrates to the $Oxy$ will contact to the surface in the three dimensional space. 
This condition expresses the $Oz$ dimension of all the $(A,B,C,D,E)$ vertices which equal to $0$ or $A_z = B_z = C_z = D_z = E_z = 0$. 
Then, 
\begin{equation*} 
\label{equa:eq01}
\begin{split}
P_z = Q_z = R_z = S_z = T_z &= 2.r_i \\
							&= \frac{a}{10}\sqrt{10(25+11\sqrt{5})}
\end{split}
\end{equation*}
\noindent It can be seen that the distance $|AF|$ is $a$ and the distance $|BF|$ is $2l_3$. Using the distance properties and squaring the results give:
\begin{equation*} 
\label{dodeca:eq1}
\begin{split}
AF^2 & = a^2 = (A_x-F_x)^2 + (A_y-F_y)^2 + (A_z-F_z)^2 \\
BF^2 & = (2l_3)^2 = (B_x-F_x)^2 + (B_y-F_y)^2 + (B_z-F_z)^2
\end{split}
\end{equation*}

\noindent Figure \ref{fig:dodecahedron2}(d) shows that $A_x = F_x = 0$, $A_y = l_1$, $B_y = l_4$, $B_x = l_3$, $A_z = B_z$. Define $l_5=A_z-F_z$, the relations of these equations are:
\begin{equation*} 
\label{dodeca:eq2}
\begin{split}
a^2 & = (F_y-l_1)^2 + l_5^2\\
(2l_3)^2 & = (F_y-l_4)^2 + l_5^2 + l_3^2
\end{split}
\end{equation*}

\noindent Solving $F_y$ and $l_5$ gives:
\begin{equation*} 
\label{dodeca:eq3}
\begin{split}
F_y & = \frac{a^2-(2l_3)^2-(l_1^2-l_3^2-l_4^2)}{2(l_4-l_1)} \\
l_5 & = \frac{1}{\sqrt{2}}\sqrt{a^2+(2l_3)^2-(F_y-l_1)^2-(F_y-l_4)^2-l_3^2}
\end{split}
\end{equation*}

\noindent From these above equations, all the vertices can be determined and stored in a $3D$ matrix. 
Path planning based on rolling contact is the rotation of this matrix through edges' contact.\\
%
%
%===========================Path Planning=============================

\noindent\uline{Path Planning}: As mentioned in the Section \ref{sec:Algorithm}, the dodecahedron can roll on the two grid types. 
The simple case has a gap at the connection between three regular pentagons as shown in Figure \ref{fig:dodecaGrid}. 
%
In this paper, we do not apply the proposed algorithm to find the path of dodecahedron with rolling on the case of overlaps between four pentagons. 
This overlap grid is complicated environment which can not guarantee to generate paths with proposed algorithm. 
%
In the case of gird with gaps (Figure \ref{fig:dodecaGrid}), attaching other five pentagons generates five $\ang{36}$ gaps from a regular pentagon. 
The six pentagons has a shape of large pentagon and the cycle of attaching more five larger pentagons will shape a grid of pentagons.\\

%
\begin{figure}[h]
\centering
	\includegraphics[width=\textwidth]{image/dodecaGrid.png}
	\caption{Initial dodecahedron grid with gaps}
	\label{fig:dodecaGrid}
\end{figure}

\noindent The rotation angle of dodecahedron on the above grid equals 
($\pi-\arccos{(\frac{-\sqrt{5}}{5})}$). 
One of the shortest paths is shown in the Figure \ref{fig:dodecaPath1} with $[3.0\ 2.25\ 0.0]$ for both the initial and goal coordinates.
One time rolling of dodecahedron generates a new dodecahedron which has a distance $d=2l_2$ from the previous coordinate. 
From the Figure \ref{fig:dodecahedron2}b, the distance $d_d$ can be determined as following.
\begin{equation*} 
\label{dodeca:eq4}
\begin{split}
l_2 & = l_1\cos{\alpha_4}\\
    & = \frac{a}{2\sin{\alpha_4}}\cos{\alpha_4}\\
    & = \frac{1}{2}a\cot{\alpha_4}\\
\rightarrow d_d & = a\cot{\alpha_4}
\end{split}
\end{equation*}

\noindent Then, the total length of the shortest path equals $15a\cot{\alpha_4}$ as shown in the Figure \ref{fig:dodecaPath1} with the red line.\\
\begin{figure}[h]
\centering
	\includegraphics[width=\textwidth]{image/dodecaPath1.pdf}
	\caption{The rolling path of dodecahedron at $[3.0\ 2.25\ 0.0]$ coordinate}
	\label{fig:dodecaPath1}
\end{figure}

\noindent Path planning through rolling contact for the case of dodecahedron is complex. According to the simulation results, implementing the proposed algorithm for other four types of platonic solids (cube, tetrahedron, octahedron, and icosahedron) is more effective than for the dodecahedron case.  Executing the algorithm is no guarantee of path-finding results with the case of overlap grid.
%%==================================================================================
%%

%%
%%






