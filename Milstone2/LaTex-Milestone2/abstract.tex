
\noindent\section{ABSTRACT}
%\textcolor{red}{250 words or less, concise summary of research conducted, results obtained, and conclusion reached}
\noindent \uline{Background}: 
Rolling geometry between two regular rigid surfaces of objects in continuous space can generate paths which is a well-known nonholonomic system. However, obtaining the path planning of a compact-closed surface on a general surface is challenging. From the rolling contact viewpoint, path-finding of convex polyhedrons has not received enough attention, and their paths on discrete surfaces have remained unexplored. \\
\noindent \uline{Aims}: The aim of this project is to develop path planning algorithms for rolling polyhedron on a surface.\\
\noindent \uline{Approaches}: 
The approach presented in this paper is based on using tree exploration technique to solve the closed-path planning disregarding orientation. 
The graph search starts from a root node then expands to different branches which indicates multiple levels of rolling.
We also focus on close-path planning for platonic solids on discrete surfaces.\\
\noindent \uline{Conclusions}: 
The results show multi-suboptimal closed paths based on rolling for the platonic solids to achieve the desired configuration on the known environments. 
The results also show successful in reaching the closed-path after combining with the point-to-point path planning technique in terms of changing polyhedrons' orientation.
Matlab simulations are carried out to demonstrate the effectiveness of the proposed path-finding algorithm for the regular polyhedra. \\ 
\noindent \uline{Implications}: its implementation can be extended to any convex polyhedrons.


%Background (2-3 sentences): difficulty in path planning of a compact-closed surface on a general surface. Convex polyhedrons rolling not receiving enough attention, and not solved
%%Aims (1 sentence): the aim of the project is to …
%Approaches (2-3): point-to-point planning disregarding orientation has been solved. So we focus on close-path planning ….
%Results (2-3): successfully complete path planning for the platonic solids
%Conclusions (1-2)
%Implications (1-2): could be extended to any convex polyhedrons


%In the modern manufacturing industry, path planning in industrial robot applications is an important step to find the shortest direction of achieving a task. Among the path planning algorithms, graph search or exploration methods has been applied for robotic in a high-dimensional space. 
%
%In the geometrical scenarios, the problem of path planning of polyhedra with rolling contact is considered. However, their rolling behaviour with returning to the initial configuration in different orientation has remained unexplored. 
%
%To tackle the problem, this study proposes a path planning method for regular platonic solids through rolling contract on a plane based on an improved tree search algorithm.
%
%The results reveal that the proposed path planning method can enhance the efficiency of the planning for regular convex polyhedra.
%
%Consequently, Matlab simulations are conducted in order to demonstrate the proposed algorithm in terms of finding the shortest path of rolling the regular platonic solids in a discrete environment.

%\textcolor{blue}{\uline{Background}: Place the question addressed in a broad context and highlight the purpose of the study.}\\
%
%
%\textcolor{blue}{\noindent\uline{Aim}: }\\
%
%\textcolor{blue}{\noindent\uline{Approach}: Methods: Describe briefly the main methods or treatments applied;}\\
%
%\textcolor{blue}{\noindent\uline{Significance}:Results: Summarize the article’s main findings;}\\ 
%
%\textcolor{blue}{\noindent\uline{Conclusion}: Indicate the main conclusions or interpretations. The abstract should be an objective representation of the article, it must not contain results which are not presented and substantiated in the main text and should not exaggerate the main conclusions.}
%
%Examples: from "2018 Path Planning of Industrial Robot - RRT"
%
%With the development of modern manufacturing industry,the application scenarios of industrial robot are becoming more and more complex. Manual programming of industrial robot requires a great deal of effort and time. \textbf{Therefore}, an autonomous path planning is an important development direction of industrial robot. 
%
%Among the path planning methods, the rapidly-exploring random tree (RRT) algorithm based on random sampling has been widely applied for a high-dimensional robotic manipulator because of its probability completeness and outstanding expansion. \textbf{However}, especially in the complex scenario, the existing RRT planning algorithms still have a low planning efficiency and some are easily fall into a local minimum. 
%
%\textbf{To tackle these problems}, this paper proposes an autonomous path planning method for the robotic manipulator based on an improved RRT algorithm. The method introduces regression mechanism to prevent over-searching configuration space. 
%\textbf{In addition}, it adopts an adaptive expansion mechanism to continuously improve reachable spatial information by refining the boundary nodes in joint space, avoiding repeatedly searching for extended nodes. 
%\textbf{Furthermore}, it avoids the unnecessary iteration of the robotic manipulator forward kinematics solution and its time-consuming collision detection in Cartesian space. The method can rapidly plan a path to a target point and can be accelerated out of a local minimum area to improve path planning efficiency. 
%
%The improved RRT algorithm proposed in this paper is simulated in a complex environment. The results reveal that the proposed algorithm can significantly improve the success rate and efficiency of the planning without losing other performance.\\





