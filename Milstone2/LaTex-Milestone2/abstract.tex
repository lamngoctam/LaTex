
\noindent\section{ABSTRACT}
\textcolor{red}{250 words or less, concise summary of research conducted, results obtained, and conclusion reached}
\textcolor{blue}{\uline{Background}: Place the question addressed in a broad context and highlight the purpose of the study.}\\

\textcolor{blue}{\noindent\uline{Aim}: }\\

\textcolor{blue}{\noindent\uline{Approach}: Methods: Describe briefly the main methods or treatments applied;}\\

\textcolor{blue}{\noindent\uline{Significance}:Results: Summarize the article’s main findings;}\\ 

\textcolor{blue}{\noindent\uline{Conclusion}: Indicate the main conclusions or interpretations. The abstract should be an objective representation of the article, it must not contain results which are not presented and substantiated in the main text and should not exaggerate the main conclusions.}

Examples: from "2018 Path Planning of Industrial Robot - RRT"

With the development of modern manufacturing industry,the application scenarios of industrial robot are becoming more and more complex. Manual programming of industrial robot requires a great deal of effort and time. \textbf{Therefore}, an autonomous path planning is an important development direction of industrial robot. 

Among the path planning methods, the rapidly-exploring random tree (RRT) algorithm based on random sampling has been widely applied for a high-dimensional robotic manipulator because of its probability completeness and outstanding expansion. \textbf{However}, especially in the complex scenario, the existing RRT planning algorithms still have a low planning efficiency and some are easily fall into a local minimum. 

\textbf{To tackle these problems}, this paper proposes an autonomous path planning method for the robotic manipulator based on an improved RRT algorithm. The method introduces regression mechanism to prevent over-searching configuration space. 
\textbf{In addition}, it adopts an adaptive expansion mechanism to continuously improve reachable spatial information by refining the boundary nodes in joint space, avoiding repeatedly searching for extended nodes. 
\textbf{Furthermore}, it avoids the unnecessary iteration of the robotic manipulator forward kinematics solution and its time-consuming collision detection in Cartesian space. The method can rapidly plan a path to a target point and can be accelerated out of a local minimum area to improve path planning efficiency. 

The improved RRT algorithm proposed in this paper is simulated in a complex environment. The results reveal that the proposed algorithm can significantly improve the success rate and efficiency of the planning without losing other performance.\\