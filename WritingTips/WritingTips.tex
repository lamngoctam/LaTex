
\documentclass{article}
\usepackage[utf8]{inputenc}


%page setting
\usepackage[left=25mm, right=25mm, top=25mm, bottom=25mm]{geometry}

%pakages
%\usepackage[colorlinks=true]{hyperref}
\usepackage{hyperref}
\usepackage{url}
\usepackage{graphicx}
\usepackage{float}
\usepackage{amsfonts}
\usepackage{amsmath}
\usepackage{amssymb}

\usepackage{enumerate}
\usepackage{fancyhdr} %footer-header
%------underline setting--------
\usepackage{ulem}

%--------Table-related commands------
\usepackage{array} %To automatically break longer lines of text within cells, define fixed-width columns
\usepackage[table,xcdraw]{xcolor}

%\usepackage{multirow}
\usepackage{tabularx} % length of table
\usepackage{caption} %space btw caption and table

\usepackage{booktabs}
% produce heavier lines as table frame (\toprule, \bottomrule) and lighter lines within a table (\midrule).

%link: https://texblog.org/2017/02/06/proper-tables-with-latex/
\newcolumntype{V}{>{\bf\centering\arraybackslash} m{0.2\linewidth} } %Repeat column type
%------------------
\usepackage{stackengine}

%--------------Tikz-------------------------------
\usepackage{import}
\usepackage{tikz}
\usepackage{tikz-3dplot}
\usepackage{subfigure}

\usetikzlibrary{shapes.geometric} %draw the flow chart

\usetikzlibrary{positioning} % https://tex.stackexchange.com/questions/94386/package-pgf-math-error-unknown-operator-o-or-of

\usetikzlibrary{automata} %for graph-automata

\usepackage{pgfplots} %for plotting data
%-----------------------------------------------
% PACKAGES for \tkzDefPoints,\tkzPolygon
%-----------------------------------------------
\usepackage{tkz-euclide}
\usepackage{siunitx} %to display angle in degree \ang{180}
\usepackage{fourier} %\widehat{A}
\usetkzobj{all}

\usepackage{verbatim} % a  drawing of a tetrahedron inscibed in a parallelepipe from https://www.overleaf.com/project/5dc97a9a3af030000156a35f
%-------------------------
%\usepackage[nottoc, notlot, notlof]{tocbibind}

%\usepackage{cite}
%\usepackage{natbib} % 
%\usepackage[numbers,sort&compress]{natbib} % sort of citation
\usepackage{pdfpages}

\usepackage{biblatex}
\addbibresource{citation.bib}



%-------------- page editors -----------------------
%\renewcommand{\baselinestretch}{1.5}
\pagestyle{fancy}
\fancyhead{}
%\fancyfoot{}
\renewcommand{\headrulewidth}{0pt}
\renewcommand{\footrulewidth}{1pt}
%------underline setting--------
\renewcommand{\ULdepth}{1.8pt}


\title{Writing Tips for PhD students}
\author{lamngoctam }
\date{November 2019}

\begin{document}

\maketitle

\section{Writing tips}
\uline{Write with Precision}: \\
- It may therefore not be unexpected that... Should be "These results suggest..."\\
- An effort was made to... should be "We tried to..."\\

\noindent\uline{Use Objective Words}: \\
- \textbf{Expressions with no clear limits}, such as: a lot, fairly, long term, quite, really, short term, slightly, somewhat, sort of,
very.\\

-\textbf{ Words of personal judgment}, such as: assuredly, beautiful, certainly, disappointing, disturbing, exquisite, fortuitous,
hopefully, inconvenient, intriguing, luckily, miraculously, nice, obviously, of course, regrettable, remarkable, sadly, surely, unfortunately.\\

- \textbf{Words that are only fillers}, such as: alright, basically, in a sense, indeed, in effect, in fact, in terms of, it goes with-
out saying, one of the things, with regard to.\\

- \textbf{Casual colorful catchwords and phrases}, such as: agree to disagree, bottom line, brute force, cutting edge, easier said than done, fell through the cracks, few and far between, food for thought, leaps and bounds, no nonsense, okay, quibble, seat of the pants, sketchy, snafu, tad,tidbit, tip of the iceberg.\\

- 

\section{References}
[1] Michael Jay Katz. From Research to Manuscript - A guide to Scientific Writing. Second Edition.


\cleardoublepage
\printbibliography
\end{document}