%
%-- link :https://tex.stackexchange.com/questions/17204/drawing-polyhedra-using-tikz-with-semi-transparent-and-shading-effect/17209#17209
%
\documentclass{minimal}
\usepackage{biblatex} 


\usepackage{tikz,tikz-3dplot}

\definecolor{cof}{RGB}{219,144,71}
\definecolor{pur}{RGB}{186,146,162}
\definecolor{greeo}{RGB}{91,173,69}
\definecolor{greet}{RGB}{52,111,72}

\tdplotsetmaincoords{70}{165}

\begin{document}
\centering
  \begin{tikzpicture}[scale=3,tdplot_main_coords]
    \coordinate (O) at (0,0,0);

%    \draw[thick,->] (0,0,0) -- (1,0,0) node[anchor=north east]{$x$};
%    \draw[thick,->] (0,0,0) -- (0,1,0) node[anchor=north west]{$y$};
%    \draw[thick,->] (0,0,0) -- (0,0,1) node[anchor=south]{$z$};

    \tdplotsetcoord{A}{1}{90}{0}    % cartesian (1,0,0)
    \tdplotsetcoord{B}{1}{90}{90}   % cartesian (0,1,0)
    \tdplotsetcoord{C}{1}{90}{180}  % cartesian (-1,0,0)
    \tdplotsetcoord{D}{1}{90}{270}  % cartesian (0,-1,0)
    \tdplotsetcoord{E}{1}{0}{0}     % cartesian (0,0,1)
    \tdplotsetcoord{F}{1}{180}{0}   % cartesian (0,0,-1)

    \draw (A) -- (B) -- (C);
    \draw (E) -- (A) -- (F);
    \draw (E) -- (B) -- (F);
    \draw (E) -- (C) -- (F);
    \draw[dashed] (C) -- (D) -- (A);
    \draw[dashed](E) -- (D) -- (F);
    \fill[cof,opacity=0.6](A) -- (B) -- (E) -- cycle;
    \fill[pur,opacity=0.6](A) -- (B) -- (F) -- cycle;
    \fill[greeo,opacity=0.6](B) -- (C) -- (E) -- cycle;
    \fill[greet,opacity=0.6](B) -- (C) -- (F) -- cycle;
  \end{tikzpicture}
  
\printbibliography
\end{document}