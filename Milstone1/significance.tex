

\section{SIGNIFICANCE}
\noindent\uline{Research gap.}
As deeply searching on the literature review, integrating tactile sensors into robot hands may improve the dexterity of robot in-hand manipulation \cite{Lei14_Teleoperation_ThumbRobotHand} \cite{Bagnell12_IntegratedSystem_AutonomousRoboticsManipulation} that may emulate human sensing.
However, there were not significant investigations into discrete differential geometry and rolling-based multifingered robotic in-hand manipulation. Specifically, discrete path planning has not been proposed for the general polyhedron in rolling contact and the cost and capacity of path planning under discrete rolling contact also need to be optimized.\\

%Besides, the important tools in differential geometry including curvature theory and Lie group theory are formed to approximate geometry attributes \cite{Lei09_Kinematic.Geometry_Circular.Surfaces, Lei09_coordinate-free_instantaneous_kinematics, Lei10_Darboux-Frame, Lei10_Geometric.Kinematics_PointContact, Lei15_PolynomialFormulation_InverseKinematics, Lei15_sliding.rolling.loci_kinematics}. 
%----------------------------------------------------------------
\noindent\uline{Research outcome.}
During the PhD program, I propose to develop a discrete contact theory based on the discrete differential geometry; then apply this tool to implement the dexterity of rolling contact into the robotic hands. 
Discrete path planning tasks will be proposed though the improvement of RRT-connect and RRT* algorithms. Then the Bellman equation will be solved for optimal path planning of the general polyhedron.
The research will apply for robot manipulation that contribute to the development of robot hands to achieve human-like capacity for in-hand manipulation. Not only the academic sector, the industrial applications are also benefited from the research. 



%\textcolor{red}{\textit{•	Include a discussion of the motivation and advantages for rolling contact for in-hand manipulation}} 