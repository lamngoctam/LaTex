\section{RESEARCH METHOD}
\noindent\uline{Proposed methodology.}
The research aims to propose a new discrete contact theory which will develop the capacity of tactile sensing in robot hands. To begin, the deep exploration of literature review on the theory of rolling contact in multifingered robot hands in both continuous and continuous spaces. Then, it is also important to build up a new mathematical model of the discrete rolling contact which can apply on the object manipulation by multifingered robot hands. To complete the proposed study, the process should be separated into three main steps: theoretical approach, simulated system, and experimental validation.\\

\noindent\uline{Theoretical approach.} The study mainly focuses on the rolling contact between robot hands and objects in discrete space. Firstly, it is crucial to explore the previous theory from continuous to discrete differential geometry as the tools to describe the rolling contact. It can be referenced from previous studies in 
\cite{Lei09_coordinate-free_instantaneous_kinematics, Lei10_Darboux-Frame, Lei15_PolynomialFormulation_InverseKinematics, Lei15_sliding.rolling.loci_kinematics, Lei10_PhDThesis}
which described the contact theory in terms of the moving frame, normal curvature, geodesic curvature, and geodesic torsion. In the next stage, specifically reviewing the path planning of regular and general polyhedron in discrete space is also proposed in the study. Finally, the discrete contact theory will be applied for the robotic in-hand manipulation.\\

\noindent\uline{Simulated system.} In the discrete path planning stage, Bellman equation is also used as the key element to build up a mathematical model. A simple task would be solving the problem of the rolling contact between a ball and a plane in discrete space. Then, the developed tasks can be applied for other objects within different geometries of polyhedron such as a cube or an octahedron. The successful of these problem solving will be analysed for point/surface contacts between an object and the robot fingers. The new platform in a simulated environment will be designed and tested by using Matlab programming and the open source ROS to solve and evaluate mathematical formulas. Once the successful applications of the system are achieved, it can be moved on to the next stage to validate the full operation in real multifingered robot hands.\\

\noindent\uline{Experimental validation.} The final stage of this study is to validate the proposed theory in terms of rolling contact in multifingered robot hands. The BarrettHand is used with the tactile sensing that can provide the tactile-array data from robot fingers and palm. The ABB IRB 120 robotic arm will be integrated with the BarrettHand to manipulate the object in discrete space.



%\textcolor{red}{•Please also review the writing for grammatical correctness (seek some assistance on this if needed).}\\


