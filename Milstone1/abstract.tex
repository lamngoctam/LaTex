
\section{ABSTRACT}
\uline{Background}: There has been much research about the establishment of rolling contact to in-hand manipulation dexterity that led to a specific moving frame method from differential geometry theory. Rolling-based plays an important role for robots with in-hand manipulation, which is considered necessary to analyse the moving object in an aspect of rolling contact. Besides, the manipulation tasks of the multifingered robot hands via tactile fingertips has been significantly considered to enhance dexterity in terms of object manipulation. Nonetheless, the discrete contact theory of discrete differential geometry has not been proposed in in-hand manipulation through rolling contact.  \\

\noindent\uline{Aim}: The aim of this project is to develop the rolling contact theory between objects in discrete space through discrete differential geometry theory. It is also important to consider discrete path planning to demonstrate the rolling contact between an object and multifingered robot hands for in-hand manipulation.\\


%The target of this project is to eliminate obstacles with in-hand manipulation by using the discrete differential geometry %\cite{Lipshutz69_ShaumOutline_DiffGeo, Carmo76_Book_Diff.Geo_CurvesSurfaces} to generate a discrete contact theory. It is also important to consider the curvature theory of smooth surfaces and the Lie group theory %\cite{Selig05_Book_Geometric_Fundamentals_Robotics} in kinematics multifingered robotic hands with rolling contact. To be demonstrated the problem of rolling contact under the discrete space, improving discrete path planning method - RRT and using Bellman equation %\cite{Bellman57_DynamicProgramming} for optimal discrete path planning %\cite{LaValle06_PlanningAlgorithm} can be effective methods in different tasks.\\


%\textcolor{red}{•Simply the aims: remove specific techniques and algorithms, and describe the broad aim of the project general terms, and in one or two sentences. Ensure that specific objectives are framed so that the aim can be achieved.}\\

\noindent\uline{Approach}: Solving the path planning task is one of the crucial stages of the research. From the literature review, there are several methods to tackle Bellman's Equation for discrete path planning problem including policy iteration, value iteration and linear programming. A discrete contact theory between an object and multifingered robot hands will be also developed by using differential geometry theory in terms of moving frame, curvature, and Lie-group theory.\\

%\textit{\textcolor{red}{which is significant methods that can be used in the study???}}\\

\noindent\uline{Significance}: Rolling-based contact may improve the dexterous ability of multifingered robot hands to arbitrarily configure or reorient manipulated objects. The improvement of the discrete contact theory based on differential geometry will be applied to robot in-hand manipulation that can contribute to the advance of industrial robotic technologies.

%According to [Leicui thesis p.41], the advantages of rolling contacts are including reduction of abrasion wear, simplification of controller, and enlargement of reachable configuration.


%\textcolor{red}{\textit{•	Include a discussion of the motivation and advantages for rolling contact for in-hand manipulation}} 