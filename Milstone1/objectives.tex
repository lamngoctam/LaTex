\section{OBJECTIVES}

\noindent The aim of this research is to generate a new mathematical model of rolling-based robotic in-hand manipulation in discrete space.  In addition, rolling contact in the multi-fingered robot hand is further considered. The specific objectives for this research project are as followed by steps below:
\begin{enumerate}[(i)]
  \item Discretized rolling contact model: An initial task of the study is to develop the geometrical framework in discrete space to form differential geometry of the rolling contact between two models (an object and robot fingers) in terms of moving frame, curvature, torsion, and the Lie-group theory.
   
  \item Discrete path planning for polyhedrons: The task is required to examine whether the path exists under solving the motion planning problems and then to generate these paths. The discrete path planning will be investigated from regular polyhedrons to general polyhedron. Discrete search algorithms are approached to locate a grid resolution and the discrete contact will be analyzed for the path planning process. 
  
  %\item Discrete  path generation: Discrete planning will be investigated by state-space models involving the distinct situation (state) and the set of possible states (space). The connection from an initial configuration to goal configuration in discrete space can generate the discrete path under rolling contact between an object and multifingered robot hands.  
  
  %When the path exists, how can I generate/find the path?
  
  \item Optimal path planning: Because path planning between the object and multifingered robot hands faces with the high dimensional environment, the optimal path planning should be considered. I also propose the analysis technique to improve both cost and capacity of the discrete path planning under rolling contact condition in terms of discrete differential geometry. 
  
  %I can find the path, how can I know whether the path is optimal or not?
  
  \item Experimental validation for dexterous robot in-hand manipulation: At the last stage of the study, Matlab and Robot Operating System (ROS) will be used to test the system in simulated environment and then ABB robot arm is integrated to BarrettHand to run the system as a physical robot.
  
\end{enumerate}



%$=>$ build up a discrete framework to formulate and solve the motion planning problems.