\documentclass[12pt]{article}

% Any percent sign marks a comment to the end of the line

% Every latex document starts with a documentclass declaration like this
% The option dvips allows for graphics, 12pt is the font size, and article
%   is the style

\usepackage[pdftex]{graphicx}
\usepackage{url}
\usepackage{hyperref}

% These are additional packages for "pdflatex", graphics, and to include
% hyperlinks inside a document.

\setlength{\oddsidemargin}{0.25in}
\setlength{\evensidemargin}{0.25in}
\setlength{\textwidth}{6.5in}
\setlength{\topmargin}{0mm}
\setlength{\textheight}{8.5in}


% These force using more of the margins that is the default style

\begin{document}

% Everything after this becomes content
% Replace the text between curly brackets with your own

\title{A Template for Your Project Proposal}
\author{Your group members names here}
\date{\today}

% You can leave out "date" and it will be added automatically for today
% You can change the "\today" date to any text you like


\maketitle

% This command causes the title to be created in the document

\section{Introduction}

% An article style is separated into sections and subsections with 
%   markup such as this.  Use \section*{Principles} for unnumbered sections.

Each working group should prepare a brief proposal describing what they want 
to do.  Please make sure that everyone participates in the discussions and
reviews the document.  The usual way to do this is to have a principle author,
but to pass it around so that everyone can comment and add or edit.  Prepare the
document using \LaTeX\/ because it is good practice and will help you learn the
basics.  However, note that Google Docs (a.k.a. Google Drive) also allows
\LaTeX\/ math symbols and is a reasonable alternative except for submissions
to journals for professional use.  There is help for \LaTeX\/ on the
class website.

Write a brief introduction in which you  outline the scope of your proposed
work. Use this space to explain why you are interested in this topic and what
you hope to learn. Connect your interest with what is currently known, and
include at least two references to related articles in the astronomical
literature.  You can use ADS  \url{http://adswww.harvard.edu/} and other links
on the class website \url{http://prancer.physics.louisville.edu/classes/308/} 
to help you find out more. An example citation is this paper
%\cite{gonzalez2012}.

One to two pages should suffice for this part, but use more if you want.  
Include figures or images if needed.  Figure \ref{fig:graph1} is an example of  how to
do it in \LaTeX.


\begin{figure}
\begin{center}
\resizebox{6in}{!}{\includegraphics*{good_abstract.png}}{}
\end{center}

\caption{The Orion Nebula, M42, recorded with the CDK20N telescope on the night
of November 1, 2011. This is a composite of three 100-second photometric images
in the Sloan g' (shown as blue), r' (shown as red), and i' (shown as green)
bands. The intensities are displayed with logarithmic compression. Click to see
the inner regions with square root compression. Highly reddened stars are
brighter in the infrared (i') and appear slightly green in these images.}
\label{fig:graph1}
\end{figure}

\section{Targets}

Identify the targets you want to study.  Define their observable characteristics
by giving their identifiers (e.g. common name if any, NGC or HD or other catalog
entry you can find with AstroCC and Simbad), celestial coordinates, optical
magnitudes, and angular size if the object is extended, that is, non-stellar.
Please select targets that we have a good likelihood of observing:
\begin{itemize}
\item Not fainter than magnitude 19 (18 is better)
\item Not larger than $0.5^\circ$, but see below
\item Above the horizon at either observatory for several hours this fall
\end{itemize}

A single 100 second exposure with the 0.5 meter telescopes will reach magnitude
18 on a clear night.  Accurate quantitative measurements require a little
brighter, or longer total accumulated exposures.  The telescopes resolve 1
arcsecond in two pixels and have a field of view of $0.6^\circ$.  Larger fields
must be mosaics of several exposures. These factors will affect your
choices.  

For planetary imaging the CDK20's can take exposures as short as 0.01 seconds. 
The longest practical single exposure is about 300 seconds, but typically we
take 100 second exposures and add them in order to make small guiding
corrections between exposures. Use AstroCC with Stellarium to assure that the
targets are observable this season.

\section{Filters, exposures, and special requests}

\subsection{Scientific rationale:}

% Please describe the scientific background of the project,
% pertinent references and previous work relevant to this 
% proposal.

\subsection{Immediate objective:}

Assuming that the best telescope for your work is one of the two 0.5 meters
(CDK20N at Moore Observatory, CDK20S at Mt. Kent), you will have a choice of
filters:  Sloan filter set (g, r, i, or z),  Johnson-Cousins (U, B, V, R, or I),
color imaging (B, G, R, or clear), and narrow band (S $[II]$, red continuum,
H$\alpha$,  O$[III]$.  Identify which filters are of interest.

A typical exposure time for a magnitude 12 star to about half saturation is 100
seconds, but it depends on the filter choice.  Based on this, estimate how many
exposures you will need, and what total time you require.  In some cases, for
example studying an eclipsing or variable star, or an exoplanet transit, you
would use only one filter and make many measurements over a night.  In others,
you may make only a few exposures in each filter, and try many different
filters.   Changing filter sets takes an operator and several minutes, but
changing filters within one set (e.g. a different Sloan filter) takes only a few
seconds.

We have other telescopes that may be available at Moore Observatory this season.
There is a wide field astrograph that has a field of view of $4^\circ$ and is a
fast $f/4$,  especially good for large nebula, comets, or surveys.  A 14-inch
(0.36 meter) Celestron  telescope can be equipped with a fast camera for
planetary imaging.  A 27-inch (0.7 meter)  corrected Dall-Kirkham is scheduled
to be be delivered to Australia this fall, although we are unsure of the actual
date it could see light yet.  

\section{Citation}
\label{sec:citations}
ABC \cite{cole1989kinematics} and \cite{sadeghi2009review} have done some interested research in rolling contact.


\newpage
\bibliographystyle{IEEEtran}{}
\bibliography{citation}


% \begin{thebibliography}{99}

% \bibitem{gonzalez2012} Jonay I. Gonz\'{a}lez Hern\'{a}ndez, 
% Pilar Ruiz-Lapuente,	
% Hugo M. Tabernero,	
% David Montes,	
% Ramon Canal,	
% Javier M\'{e}ndez	
% and Luigi R. Bedin,
% {No surviving evolved companions of the progenitor of SN1006},
% Nature, {\bf 489}, 533-536 (2012).

% \end{thebibliography}



\end{document}