
\section{Table Notes}


\begin{table}[ht]
\centering
\caption{Thicker horizontal lines above and below the table.}
\begin{tabular}{lcc}
\toprule
&Treatment A&Treatment B\\ 
\hline
%\midrule
John Smith&1&2\\
Jane Doe&--&3\\
Mary Johnson&4&5\\
\bottomrule
\end{tabular}
\end{table}%


Sometimes very long tables must be presented which may also go over pages. \\


\begin{table}[ht!]
     \centering
     \caption{Caption text} 
     \begin{tabular}{|p{5cm}|p{5cm}|}
        \toprule 	
	\bfseries{Resources} &\bfseries{Provider}\\ \hline
	Robot Operating Software & Open Source\\
	MatLab                  & Curtin University\\
	ABB Robotic Arm         & Curtin University\\
	Computer and Printing   & Curtin University\\
	\bottomrule
      \end{tabular}
    \end{table}
    
\begin{table}[h]
 \centering
     \caption{Long table} 
\begin{tabularx}{\textwidth}{X|l}
  \textbf{Symptom} & \textbf{Metric} \\
\hline
Class that & ATFD is more than a few\\
Class that & WMC is high\\
Class that & TCC is low\\
\end{tabularx}
\end{table}


\begin{table}[ht]
\caption{Repetition of custom-defined column type.}
\begin{center}
\begin{tabular}{*{3}{V}}
\hline
aaa&bbb&ccc\\
\hline
aaa&bbb&ccc\\
aaa&bbb&ccc\\
\hline
\end{tabular}
\end{center}
\end{table}
